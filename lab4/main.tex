 \documentclass[5]{article}
\usepackage[utf8]{inputenc}
\usepackage{hyperref} 

\usepackage[T1]{fontenc}
\usepackage[polish]{babel}

\title{Laboratorium 4}
\author{Piotr Witek}
\date{14 kwietnia 2021}

\usepackage{natbib}
\usepackage{graphicx}
\usepackage{geometry}
\usepackage{tabularx}
\usepackage{array}
\usepackage{amsmath}

\begin{document}

\newgeometry{tmargin=2cm, bmargin=2cm, lmargin=2.5cm, rmargin=2.5cm}

\maketitle

\section{Zadania}

\subsection{Obliczyć $\mathrm{I}={ }_{0} \int^{1} 1 /(1+\mathrm{x}) \mathrm{dx}$ wg wzoru prostokątów, trapezów i wzoru Simpsona (zwykłego i złożonego $\mathrm{n}=3,5$ ). Porównać wyniki i blędy. }

Dokładna wartość całki wynosi:
$$
\int_{0}^{1} 1 /(1+x) d x=\left.\ln (x+1)\right|_{0} ^{1}=\ln 2 \approx 0.69314718056
$$
\newline

Całkowanie metodą prostokatów na zadanym przedziale [a,b] polega na obliczeniu następującej sumy:
$$
I_{n}=\sum_{i=1}^{n} f\left(c_{i}\right) * \frac{|b-a|}{n}
$$
gdzie: $c_{i}$ to środek każdego z n podprzedziałów przedziału $[a,b]$
Całka jest więc równa sumie pól prostokątów o szerokości każdego podprzedziału i wysokości równej wartości funkcji całkowanej w środku tego przedziału.
\newline

Całkowanie metodą trapezów polega na policzeniu poniższej sumy:
$$
I_{n}=\sum_{i=2}^{n} \frac{h}{2}\left(f\left(c_{i-1}\right)+f\left(c_{i}\right)\right)
$$
gdzie: $c_{i}$ to każdy z n punktów, powstałych poprzez podział przedziału [a,b] na $n-1$ przedziałów. Punkty oddalone są od siebie o $h$
\newline

Całkowanie metoda Simpsona polega na policzeniu poniższej sumy:
$$
I_{n}=\sum_{i=3}^{n} \frac{h}{3}\left(f\left(c_{i}\right)+4 f\left(c_{i-1}\right)+f\left(c_{i-2}\right)\right)
$$
gdzie: $c_{i}$ to każdy z n punktów, gdzie $n=2k+1$, powstałych poprzez podział przedziału [a,b] na $n-1$ równych podprzedziałów.
\newline


\begin{enumerate}
    \item Zwykła kwadratura prostokatów jest tożsama ze złożoną kwadraturą prostokątów dla $n=1$
    \item Zwykla kwadratura trapezów jest tożsama ze złożoną kwadraturą trapezów dla $n=2$
    \item Zwykla kwadratura Simpsona jest tożsama ze złożoną kwadraturą Simpsona dla $n=3(k=1)$
\end{enumerate}

Obliczenia powyższych kwadratur wykonałem przy pomocy programu w Pythonie, któego kod znajduje się poniżej.

\begin{verbatim}
import numpy as np
from math import log

def f(x):
    return (1/(x+1))

def rectangle():
    for steps in [1,3,5]:
        length = 1
        xs = np.linspace(0, length, num=steps+1, endpoint=False)
        Int = sum(map(lambda x: f(x)*length/steps, xs[1::]))

        print("n =", steps, Int, "bład bezwględny:", Int - log(2))

def trapez():
    for steps in [2,3,5]:
        length = 1
        h = length/(steps-1)
        xs = np.linspace(0, length, num=steps, endpoint=True)
        Int = sum(map(lambda x: (f(x)+f(x-h))*h/2, xs[1::]))

        print("n =", steps, Int, "bład bezwględny:", Int - log(2))

def simpson():
    for steps in [3,5]:
        length = 1
        h = length/(steps-1)
        xs = np.linspace(0, 0+length, num=(steps//2)+1, endpoint=True)
        Int = sum(map(lambda x: (f(x)+4*f(x-h)+f(x-2*h))*(h/3), xs[1::]))

        print("n =", steps, Int, "bład bezwględny:", Int - log(2))
\end{verbatim}


\newline

Otrzymano wyniki (odpowiednio dla metod prostokatów, trapezów, Simpsona):

\begin{verbatim}
n = 1 0.6666666666666666 bład bezwględny: -0.026480513893278657
n = 3 0.6793650793650794 bład bezwględny: -0.013782101194865892
n = 5 0.6838528138528138 bład bezwględny: -0.009294366707131463
---------
n = 2 0.75 bład bezwględny: 0.056852819440054714
n = 3 0.7083333333333333 bład bezwględny: 0.015186152773387973
n = 5 0.6970238095238095 bład bezwględny: 0.0038766289638642037
---------
n = 3 0.6944444444444443 bład bezwględny: 0.0012972638844990225
n = 5 0.6932539682539682 bład bezwględny: 0.0001067876940229473
\end{verbatim}


\subsection{Obliczyć całkę $\mathrm{I}={ }_{-1} \int^{1} 1 /\left(1+\mathrm{x}^{2}\right)$ dx korzystając $\mathrm{z}$ wielomianów ortogonalnych (np. Legendre'a) dla $\mathrm{n}=8$.}

Dokładna wartość naszej całki wynosi:
$$
I=\int_{-1}^{1} 1 /\left(1+x^{2}\right) d x=\left.\operatorname{arctg}(x)\right|_{-1} ^{1}=\frac{\pi}{2}
$$
\newline

Wykorzystuje wielomiany Legendre'a, które zadane są wzorem rekurencyjnym:
$$
\begin{array}{c}
(n+1) P_{n+1}(x)=(2 n+1) x P_{n}(x)-n P_{n-1}(x) \\
P_{0}(x)=1 \\
P_{1}(x)=x
\end{array}
$$
Są one ortogonalne na przedziale całkowania $[-1,1]$ z wagą $w(x)=1$.
\newline

Całkę będziemy przybliżać całką wielomianu aproksymującego szukaną funkcję, w naszym przypadku funkcja aproksymująca będzie mieć postać:
$$
F(x)=\sum_{i=0}^{2} c_{i} P_{i}(x), \quad x \in[-1,1]
$$
Wyznaczam współczynniki $c_{i}$ ze wzoru:
$$
c_{i}=\frac{\int_{-1}^{1} f(x) L_{i}(t) d x}{\int_{-1}^{1} L_{i}^{2}(x) d x}
$$
\newline
Następnie przy pomocy programu napisanego w języku Python wygenerowałem wielomiany Legendre'a (def genLegendre), wyznaczyłem współczynniki $c_{i}$ (def approxF). Kod poniżej.


\begin{verbatim}
import numpy as np
from scipy.integrate import quadrature

def f(x):
    return 1/(1+x**2)


def genLegendre(n):
    p = []
    p.append(np.poly1d([1]))
    p.append(np.poly1d([1,0]))
    x = np.poly1d([1,0])
    for i in range(1,n):
        p.append((2*i+1)/(i+1)*p[i]*x-(i/(i+1))*p[i-1])
    return p

def approxF(n):
    L = genLegendre(n)
    c = []
    fA = []
    fA.append(np.poly1d([0]))
    for i in range(0,n):
        c.append(quadrature((lambda x: f(x)*L[i](x)), -1,1)[0]/quadrature((lambda x: L[i](x)**2), -1, 1)[0])
        fA.append(c[i]*L[i])
    return sum(fA)

def Int(n):
    i = approxF(n).integ()
    return i(1) - i(-1)
\end{verbatim}

Dzięki funkcji genLegendre() otrzymuję funkcję aproksymującą, której całka będzie przybliżeniem całki, którą mam obliczyć. Wynik programu:

\begin{verbatim}
Wartość obliczona:  1.5707963251251216 błąd bezwględny:  -1.6697749849470256e-09
\end{verbatim}

\section{Zadania domowe}

\subsection{Obliczyć całkę ${ }_{0} \int^{1} 1 /\left(1+\mathrm{x}^{2}\right)$ dx korzystając ze wzoru prostokątów, trapezów $\mathrm{i}$ wzoru Simpsona dla $\mathrm{h}=0.1$}

Dokładna wartość całki:
$$
\int_{0}^{1} \frac{1}{\left(1+x^{2}\right)}=\tan^{-1}(x) = \frac{\pi}{4} \approx 1.5707963267948966192
$$

Analogicznie do zadania 1, po niewielkich modyfikacjach, korzystam z przygotowanego wczesniej programu:

\begin{verbatim}
import numpy as np
from math import log

def f(x):
    return (1/(1+x**2))

def rectangle():
    length = 1
    h=0.1
    xs = np.linspace(0, 1, num=10, endpoint=False)
    Int = sum(map(lambda x: f(x)*h, xs[1::]))

    print("wartość: ", Int, "bład bezwględny: ", Int - np.pi/4)

def trapez():

    length = 1
    h = 0.1
    xs = np.linspace(0, 1, num=11, endpoint=True)
    Int = sum(map(lambda x: (f(x)+f(x-h))*h/2, xs[1::]))

    print("wartość: ", Int, "bład bezwględny: ", Int - np.pi/4)

def simpson():
    length = 1
    h = 0.1
    xs = np.linspace(0, 1, num=6, endpoint=True)
    Int = sum(map(lambda x: (f(x)+4*f(x-h)+f(x-2*h))*(h/3), xs[1::]))

    print("wartość: ", Int, "bład bezwględny: ", Int - np.pi/4)

\end{verbatim}

Otrzymuję następujące wyniki (odpowiednio dla metod prostokątów, trapezów i Simpsona):

\begin{verbatim}
wartość:  0.7099814972267896 bład bezwględny:  -0.07541666617065867
---------
wartość:  0.7849814972267897 bład bezwględny:  -0.00041666617065860834
---------
wartość:  0.7853981534848038 bład bezwględny:  -9.912644483023314e-09
\end{verbatim}



\section{Bibliografia}

\begin{enumerate}
  \item \url{http://home.agh.edu.pl/~funika/mownit/lab4/calkowanie.pdf}
  \item \url{https://eduinf.waw.pl/inf/alg/004_int/0004.php}
\end{enumerate}

\end{document}
